% The paper on using Luau telemetry to measure the effectiveness of type error reporting.

\documentclass[
  acmsmall,
  review,
%  anonymous,
]{acmart}

%\settopmatter{printfolios=true,printccs=false,printacmref=false}

\overfullrule=1mm
\citestyle{acmauthoryear}
%\setcitestyle{round}

\usepackage{alltt}
% \usepackage{amssymb} -- already loaded by acmart
\usepackage{calc}
\usepackage{cleveref}
\usepackage{listings}
\usepackage{mathpartir}
\usepackage{pifont}
\usepackage{tikz}
\usepackage{wrapfig}
\usepackage{xcolor}
\usetikzlibrary{shapes.geometric}

\begin{document}

\title{Millions of Type Errors}
\subtitle{Using Telemetry To Measure The User Experience Of Type Error Reporting}

% Alphabetical order for authors?

\author{Ben Greenman}
\orcid{0000-0001-7078-9287}
\affiliation{%
  \institution{Brown University}
  \city{Providence}
  \state{Rhode Island}
  \country{USA}
}
\email{benjaminlgreenman@gmail.com}

\author{Alan Jeffrey}
\orcid{0000-0001-6342-0318}
\affiliation{%
  \institution{Roblox}
  \city{San Mateo}
  \state{California}
  \country{USA}
}
\email{}

\author{Shriram Krishnamurthi}
\orcid{0000-0001-5184-1975}
\affiliation{%
  \institution{Brown University}
  \city{Providence}
  \state{Rhode Island}
  \country{USA}
}
\email{shriram@brown.edu}

\author{Mitesh Shah}
\orcid{TODO}
\affiliation{%
  \institution{Roblox}
  \city{San Mateo}
  \state{California}
  \country{USA}
}
\email{}

%\renewcommand{\shortauthors}{...}

%%
%% The abstract is a short summary of the work to be presented in the
%% article.
\begin{abstract}
  \anon{Roblox Studio} is a tool which places programming in the hands of
  millions of creators, ranging from high school students to professional
  development studios. It has the ability to report telemetry data,
  which allows large-scale measurement of the creator experience. In
  this paper, we discuss one use of telemetry, to measure the experience
  of type error reporting.
\end{abstract}

\newcommand{\code}[1]{\texttt{#1}}
\newcommand{\FILL}{\textbf{FILL}}

%%
%% The code below is generated by the tool at http://dl.acm.org/ccs.cfm.
%% Please copy and paste the code instead of the example below.

\begin{CCSXML}
<ccs2012>
<concept>
<concept_id>10011007.10011006.10011039.10011311</concept_id>
<concept_desc>Software and its engineering~Semantics</concept_desc>
<concept_significance>500</concept_significance>
</concept>
<concept>
<concept_id>10011007.10011006.10011008.10011024.10011032</concept_id>
<concept_desc>Software and its engineering~Constraints</concept_desc>
<concept_significance>100</concept_significance>
</concept>
<concept>
<concept_id>10011007.10011006.10011008.10011009.10011012</concept_id>
<concept_desc>Software and its engineering~Functional languages</concept_desc>
<concept_significance>100</concept_significance>
</concept>
</ccs2012>
\end{CCSXML}

\ccsdesc[500]{Software and its engineering~Semantics}
\ccsdesc[100]{Software and its engineering~Constraints}
\ccsdesc[100]{Software and its engineering~Functional languages}

\keywords{types, gradual typing, telemetry, user study, large-scale study}

\maketitle

\section{Introduction}
\label{s:introduction}

\anon{Roblox} is a platform for \anon{shared virtual experiences},
with 56~million Daily Active Users, and 49~billion hours of engagement in
2022~\anon[(ANONYMIZED CITATION)]{\cite{roblox-quarterly-results}}.
There are \textbf{XX}~million creators using \anon{Roblox Studio},
and \textbf{YY}~million creations.

\anon{Roblox experiences} are scripted using the 
\anon{Luau} programming language~\anon[(ANONYMIZED CITATION)]{\cite{luau-lang.org}},
an extension of \anon{Lua~5.1~\cite{lua}}.
The main extension is the addition of a static type system, which uses
type inference to synthesize types for user code. These types
are used primarily in type-driven tooling such as autocomplete
and API documentation~\anon[(ANONYMIZED CITATION)]{\cite{luau-autocomplete}},
but creators can also opt in to receiving type error reports.

As discussed in~\anon[(ANONYMIZED CITATION)]{\cite{bfj-hatra-2021}},
the goals of the \anon{Luau} type system are rather different from
a traditional type system, which focuses on compilation and memory safety.
\anon{Luau} has a very heterogeneous user community, ranging from
students in code camps to professional development studios. These
creators have quite different needs, with different emphases on
enabling rapid creation and ensuring software quality.

In this paper, we investigate methods for measuring the effectiveness
of the \anon{Luau} type system in development of \anon{Roblox} scripts.
In comparison to prior work~(\cref{s:related}), which is either small in scale
or depends on personally identifiable information~(PII),
we performed a large-scale study using pseudonymized \emph{telemetry}.

\anon{Roblox Studio} has a telemetry system, which is used to gauge
the effectiveness of creation features. This system stochastically
determines which sessions should report telemetry, and for those
sessions, reports telemetry records back with a summary of the
session. In the case of this study, the telemetry includes data on the
number of errors at various levels of granularity: in the current edit
region, in the current file, and in every file which was type
checked.

The telemetry data we analyzed does not contain any PII:
no source code;
no source code locations;
no error messages (which may contain source code);
no record of the creator's identity, locale, or IP address;
and no information about what creation the data came from.
Telemetry records are correlated by session, using a pseudonymized
session identifier.

Most users of \anon{Roblox Studio} do not opt in to type error
reporting, and so they do not see the ``squiggly underlining'' that
indicates a type error site. Nonetheless, the type inference system
still runs (since it drives autocomplete and other type-based tools) and
so we can record which type errors would have been reported had the
user enabled type error reporting. As a result, we can investigate
which type errors are  fixed by users, even if they did not opt in to
type error reporting.

With this telemetry data, we investigate research questions about
the adoption and benefits of type analysis.
First, to learn how many creators use type analysis of any sort
and to estimate whether they pay attention to the results (\FILL{} really, attention?).
Once in type-analysis mode, do creators stay, or revert to no-check?
Second, what errors do creators face and how do they respond.
Are the error highlights helpful for removing the error?
Is there any indication that type analysis improves the quality of the development experience?

This paper is the first to use telemetry as a mechanism for
large-scale measurement of the effectiveness of type error reporting.
Our data captures \textbf{XX} sessions, for a total of \textbf{YY}
hours and \textbf{ZZ} type errors.

Telemetry cannot replace user studies, as there is no way to measure
creator sentiment, but provide complementary data at scale.

\paragraph{Contributions}
\begin{itemize}
  \item
    Design of a low-overhead, (black-box / PII-safe / impersonal)
    telemetry method, that other
    researchers can build on.

  \item
    Lessons from millions of type errors about
    the adoption of strict type analysis,
    the usefulness of type errors,
    and \FILL{}.
    These findings are especially important for the
    gradual typing, success typing, and semantic subtyping communities.

\end{itemize}


\section{\anon{Roblox} Context}
% https://create.roblox.com/docs/scripting/luau

FILL table = object

\begin{figure}
  \anon{
    \includegraphics[width=.45\textwidth]{img/roblox-studio.png}
    \includegraphics[width=.45\textwidth]{img/roblox-studio-ide.png}
  }
  \caption{\anon{Roblox Studio 3D creation} tools (left) and IDE (right)}
  \label{fig:roblox-studio}
\end{figure}
      
Creators of \anon{Roblox experiences} use \anon{Roblox Studio},
which combines \anon{3D creation} tools as well as an Integrated
Developer Environment (IDE), as seen in Fig.~\ref{fig:roblox-studio}.
The IDE includes an optional ``Script Analysis'' widget, which
reports syntax errors, type errors, and problems identified by
lint tools. The script editor also (optionally) highlights
the location in code where reported errors occur.

To opt in to type error reporting, creators set a \emph{mode}
for each script, which is one of:
\begin{itemize}
\item \emph{nocheck}: only syntax errors are reported,
\item \emph{nonstrict}: all syntax errors, and a subset of ``high probability'' type errors, are reported, or
\item \emph{strict}: all syntax errors and type errors, are reported.
\end{itemize}
As an example of nonstrict mode, the following program only reports one error:
\begin{verbatim}
--!nonstrict
local x = { p = 5, q = nil }
if condition then x.q = 7 end
local y = x.p + x.q --> no type error
local z = x.r       --> "Key 'r' not found in table 'x'"
\end{verbatim}
but in strict mode it reports two:
\begin{verbatim}
--!strict
local x = { p = 5, q = nil }
if condition then x.q = 7 end
local y = x.p + x.q --> "Type 'nil' could not be converted into 'number'"
local z = x.r       --> "Key 'r' not found in table 'x'"
\end{verbatim}
In cases like this, where it is undecidable whether there will be a run-time error,
strict mode errs on the side of reporting an error, and nonstrict mode errs on
the side of suppressing the error.

Both modes report the \verb|Key 'r' not found in table 'x'| error --
misspellings of property names are common enough to report in both
strict and nonstrict mode. See~\anon[ANONYMIZED CITATION]{\cite{bfj-hatra-2021}}
for a more detailed discussion of the rationale for strict and nonstrict mode.

Both modes are opt-in. Creators have the option to make nonstrict mode
the default rather than nocheck mode.

Even in nocheck mode, \anon{Roblox Studio} performs type inference, since
the results are needed by type-directed tooling such as autocomplete and
API documentation. This behind-the-scenes typechecking is always performed
in strict mode, since it is important that the inferred types be as precise
as possible. The type errors produced by this pass are always discarded,
so the verbosity of strict mode is not an issue.

Since this pass is always performed in strict mode, we refer to it as
\emph{forced strict} mode. Its main use is in autocomplete, so forced
strict mode is triggered on every keystroke

Scripts come in two favors: \emph{module scripts} and \emph{non-module
scripts}.  Module scripts provide reusable libraries, which may be
\emph{required} by other scripts. Since module scripts can require
other module script, modules form a graph (though we consider it to be
an error in strict mode if the graph is cyclic, and remove edges to
make it acyclic).

When typechecking is performed for script analysis, any script that
has been modified is marked as dirty, then any script that is dirty,
or which transitively requires a dirty module, is typechecked. More
commonly, when typechecking is performed for autocomplete, we only
need to typecheck the current script, since it is the only dirty
script, and nothing it requires can transitively require it, since we
have broken cycles.

The state of the world in a \anon{Roblox} experience is captured by
the \emph{data model}, which is a tree of \emph{instances}, such as
parts, models, meshes, effects, lighting, audio assets, and physics
constraints such as forces, springs and joints.

While an experience is under development, it is typical for the data
model to be edited (for example instances to be added, deleted, moved
or renamed). Since the initial shape of the data model tree is reflected in
the type system, it is possible for these edits to introduce type errors.

\begin{table}
  \caption{Selected Error Labels}
  %% from type analysis, but they're not all really type errors
  %% TODO review selection after analyzing the full data
  \label{t:type-error-labels}

  %% FILL examples for each error?
  \begin{tabular}{ll}
    Label & Interpretation \\\midrule
    \code{CodeTooComplex} & Type analysis failed, cannot understand the code \\
    \code{UnificationTooComplex} & Type analysis failed, unification solver hit a limit \\
    \code{SyntaxError} & Basic parse error, e.g., \code{for if end} \\

    \code{IncorrectGenericParameterCount} & Arity mismatch for a generic type \\
    \code{CountMismatch} & \FILL{} \\
    \code{UnknownProperty} & Referenced an invalid field or method  \\
    \code{OnlyTablesCanHaveMethods} & Tried to attach a method to a non-table \\
    \code{CannotCallNonFunction} & Called a value that is not a function \\
    \code{TypesAreUnrelated} & Failed to cast, unify, or check subtyping \\
    \code{TypeMismatch} & Generic label for other type errors \\
    \code{GenericError}
    & Generic label for other errors, e.g., declaring a \\
    & supertype that is not a class type
    %% or iterated over a table without specifying how


%    \code{ExtraInformation} & Follow-on error that provides additional information for an error at the same source location. \\
%    UnknownSymbol &  \\
%    NotATable &  \\
%    CannotExtendTable &  \\
%    DuplicateTypeDefinition &  \\
%    FunctionDoesNotTakeSelf &  \\
%    FunctionRequiresSelf &  \\
%    OccursCheckFailed &  \\
%    UnknownRequire &  \\
%    UnknownPropButFoundLikeProp &  \\
%    InternalError &  \\
%    DeprecatedApiUsed &  \\
%    ModuleHasCyclicDependency &  \\
%    IllegalRequire &  \\
%    FunctionExitsWithoutReturning &  \\
%    DuplicateGenericParameter &  \\
%    CannotInferBinaryOperation &  \\
%    MissingProperties &  \\
%    SwappedGenericTypeParameter &  \\
%    OptionalValueAccess &  \\
%    MissingUnionProperty &  \\
%    NormalizationTooComplex &  \\
%    TypePackMismatch &  \\
%    DynamicPropertyLookupOnClassesUnsafe &  \\)
  \end{tabular}
\end{table}

\section{Telemetry Design}

No PII whatsoever

Random session selection

Random events during session, except module switch.

No access to other key events: save, exit.
Don't want access to run, publish.

\url{https://docs.google.com/document/d/1DnKvw8x1jy0EWCbBSM8neKULWz7WAg1OauKy7qqYqkI/edit}




\section{Predictions}
%%bg: merge with prev section, on telemetry design?

nocheck forcestrict keep growing

nonstrict te tend to zero, fixing your program should fix the errors, no false positives

nonstrict fs unclear


\section{Results}
\label{s:data}

\begin{figure}[t]
  \includegraphics{img/row-distribution.pdf}
  \Description{TBD: histogram with about 100 records per hour except for a 600-record spike near the end of Jan 12th.}
  \caption{Telemetry records per hour. Each tick on the $x$-axis marks the start of a new day in California.}
  \label{f:records-per-hour}
\end{figure}

\Cref{f:records-per-hour} shows when data arrived across the whole dataset.

\begin{table}[t]
  \caption{Dataset overview}
  \label{t:dataset-overview}
\begin{tabular}{rl}
 5,108 & total logs \\
       & 4,574 nocheck + 510 nonstrict + 24 strict \\
 182,247 & total forced strict type errors \\
       & {113,274 in module, 6,425 in edit regions} \\
  1,521 & total type errors \\
       & {1,485 in module, 331 in edit regions}
  \\[2ex]

  1,146 & sessions \\
  & \begin{tabular}{rl}
      1,143 & single mode = 1046 nocheck + 94 nonstrict + 3 strict \\
      7 & multi mode projects \\
      3 & mode upgrades \\
      4 & mode downgrades
    \end{tabular}
\end{tabular}
\end{table}


\section{Interpretation}

Error 1000 = type mismatch (subtyping etc);
1001 = unknown symbol;
\ldots

% https://github.com/Roblox/luau/blob/master/Analysis/include/Luau/Error.h


\begin{verbatim}
 require(foobar)
  foobar is a module script in the data model
 require(foo.bar.b)
  - could be edit in progress
  - could be renamed module
  - 
\end{verbatim}



\section{Related Work}
\label{s:related}

\paragraph{Research on Errors}

Mind your language~\cite{mfk-onward-2011}.


\paragraph{Telemetry}

Transparent telemetry: explain what it is and contrast our approach.
What essential, non-transparent things did we collect?

TT =
no user ID, no machine ID,
no time-ordered traces,
public process to decide what to collect

https://research.swtch.com/telemetry-intro


\section{Discussion}
\label{s:conclusion}
\label{s:discussion}



\begin{acks}
  TBD

Greenman was supported by
  \grantsponsor{NSF}{NSF}{https://www.nsf.gov} grant
 \href{"https://www.nsf.gov/awardsearch/showAward?AWD_ID=2030859"}{\grantnum{NSF}{CCF 2030859}}
  to the CRA for the \href{https://cifellows2020.org}{CIFellows} project.
\end{acks}

\bibliographystyle{ACM-Reference-Format}
\bibliography{bib}

\end{document}
\endinput
